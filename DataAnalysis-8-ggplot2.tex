\documentclass[]{article}
\usepackage{lmodern}
\usepackage{amssymb,amsmath}
\usepackage{ifxetex,ifluatex}
\usepackage{fixltx2e} % provides \textsubscript
\ifnum 0\ifxetex 1\fi\ifluatex 1\fi=0 % if pdftex
  \usepackage[T1]{fontenc}
  \usepackage[utf8]{inputenc}
\else % if luatex or xelatex
  \ifxetex
    \usepackage{mathspec}
  \else
    \usepackage{fontspec}
  \fi
  \defaultfontfeatures{Ligatures=TeX,Scale=MatchLowercase}
\fi
% use upquote if available, for straight quotes in verbatim environments
\IfFileExists{upquote.sty}{\usepackage{upquote}}{}
% use microtype if available
\IfFileExists{microtype.sty}{%
\usepackage{microtype}
\UseMicrotypeSet[protrusion]{basicmath} % disable protrusion for tt fonts
}{}
\usepackage[margin=1in]{geometry}
\usepackage{hyperref}
\hypersetup{unicode=true,
            pdfborder={0 0 0},
            breaklinks=true}
\urlstyle{same}  % don't use monospace font for urls
\usepackage{color}
\usepackage{fancyvrb}
\newcommand{\VerbBar}{|}
\newcommand{\VERB}{\Verb[commandchars=\\\{\}]}
\DefineVerbatimEnvironment{Highlighting}{Verbatim}{commandchars=\\\{\}}
% Add ',fontsize=\small' for more characters per line
\usepackage{framed}
\definecolor{shadecolor}{RGB}{248,248,248}
\newenvironment{Shaded}{\begin{snugshade}}{\end{snugshade}}
\newcommand{\KeywordTok}[1]{\textcolor[rgb]{0.13,0.29,0.53}{\textbf{#1}}}
\newcommand{\DataTypeTok}[1]{\textcolor[rgb]{0.13,0.29,0.53}{#1}}
\newcommand{\DecValTok}[1]{\textcolor[rgb]{0.00,0.00,0.81}{#1}}
\newcommand{\BaseNTok}[1]{\textcolor[rgb]{0.00,0.00,0.81}{#1}}
\newcommand{\FloatTok}[1]{\textcolor[rgb]{0.00,0.00,0.81}{#1}}
\newcommand{\ConstantTok}[1]{\textcolor[rgb]{0.00,0.00,0.00}{#1}}
\newcommand{\CharTok}[1]{\textcolor[rgb]{0.31,0.60,0.02}{#1}}
\newcommand{\SpecialCharTok}[1]{\textcolor[rgb]{0.00,0.00,0.00}{#1}}
\newcommand{\StringTok}[1]{\textcolor[rgb]{0.31,0.60,0.02}{#1}}
\newcommand{\VerbatimStringTok}[1]{\textcolor[rgb]{0.31,0.60,0.02}{#1}}
\newcommand{\SpecialStringTok}[1]{\textcolor[rgb]{0.31,0.60,0.02}{#1}}
\newcommand{\ImportTok}[1]{#1}
\newcommand{\CommentTok}[1]{\textcolor[rgb]{0.56,0.35,0.01}{\textit{#1}}}
\newcommand{\DocumentationTok}[1]{\textcolor[rgb]{0.56,0.35,0.01}{\textbf{\textit{#1}}}}
\newcommand{\AnnotationTok}[1]{\textcolor[rgb]{0.56,0.35,0.01}{\textbf{\textit{#1}}}}
\newcommand{\CommentVarTok}[1]{\textcolor[rgb]{0.56,0.35,0.01}{\textbf{\textit{#1}}}}
\newcommand{\OtherTok}[1]{\textcolor[rgb]{0.56,0.35,0.01}{#1}}
\newcommand{\FunctionTok}[1]{\textcolor[rgb]{0.00,0.00,0.00}{#1}}
\newcommand{\VariableTok}[1]{\textcolor[rgb]{0.00,0.00,0.00}{#1}}
\newcommand{\ControlFlowTok}[1]{\textcolor[rgb]{0.13,0.29,0.53}{\textbf{#1}}}
\newcommand{\OperatorTok}[1]{\textcolor[rgb]{0.81,0.36,0.00}{\textbf{#1}}}
\newcommand{\BuiltInTok}[1]{#1}
\newcommand{\ExtensionTok}[1]{#1}
\newcommand{\PreprocessorTok}[1]{\textcolor[rgb]{0.56,0.35,0.01}{\textit{#1}}}
\newcommand{\AttributeTok}[1]{\textcolor[rgb]{0.77,0.63,0.00}{#1}}
\newcommand{\RegionMarkerTok}[1]{#1}
\newcommand{\InformationTok}[1]{\textcolor[rgb]{0.56,0.35,0.01}{\textbf{\textit{#1}}}}
\newcommand{\WarningTok}[1]{\textcolor[rgb]{0.56,0.35,0.01}{\textbf{\textit{#1}}}}
\newcommand{\AlertTok}[1]{\textcolor[rgb]{0.94,0.16,0.16}{#1}}
\newcommand{\ErrorTok}[1]{\textcolor[rgb]{0.64,0.00,0.00}{\textbf{#1}}}
\newcommand{\NormalTok}[1]{#1}
\usepackage{graphicx,grffile}
\makeatletter
\def\maxwidth{\ifdim\Gin@nat@width>\linewidth\linewidth\else\Gin@nat@width\fi}
\def\maxheight{\ifdim\Gin@nat@height>\textheight\textheight\else\Gin@nat@height\fi}
\makeatother
% Scale images if necessary, so that they will not overflow the page
% margins by default, and it is still possible to overwrite the defaults
% using explicit options in \includegraphics[width, height, ...]{}
\setkeys{Gin}{width=\maxwidth,height=\maxheight,keepaspectratio}
\IfFileExists{parskip.sty}{%
\usepackage{parskip}
}{% else
\setlength{\parindent}{0pt}
\setlength{\parskip}{6pt plus 2pt minus 1pt}
}
\setlength{\emergencystretch}{3em}  % prevent overfull lines
\providecommand{\tightlist}{%
  \setlength{\itemsep}{0pt}\setlength{\parskip}{0pt}}
\setcounter{secnumdepth}{0}
% Redefines (sub)paragraphs to behave more like sections
\ifx\paragraph\undefined\else
\let\oldparagraph\paragraph
\renewcommand{\paragraph}[1]{\oldparagraph{#1}\mbox{}}
\fi
\ifx\subparagraph\undefined\else
\let\oldsubparagraph\subparagraph
\renewcommand{\subparagraph}[1]{\oldsubparagraph{#1}\mbox{}}
\fi

%%% Use protect on footnotes to avoid problems with footnotes in titles
\let\rmarkdownfootnote\footnote%
\def\footnote{\protect\rmarkdownfootnote}

%%% Change title format to be more compact
\usepackage{titling}

% Create subtitle command for use in maketitle
\newcommand{\subtitle}[1]{
  \posttitle{
    \begin{center}\large#1\end{center}
    }
}

\setlength{\droptitle}{-2em}
  \title{}
  \pretitle{\vspace{\droptitle}}
  \posttitle{}
  \author{}
  \preauthor{}\postauthor{}
  \date{}
  \predate{}\postdate{}

\usepackage{xeCJK}
\setCJKmainfont{微软雅黑}  % 字体可以更换
\setmainfont{Georgia} % 設定英文字型
\setromanfont{Georgia} % 字型
\setmonofont{Courier New}

\begin{document}

{
\setcounter{tocdepth}{2}
\tableofcontents
}
\section{利用ggplot2绘图}\label{ggplot2}

\begin{center}\rule{0.5\linewidth}{\linethickness}\end{center}

\subsection{为什么使用ggplot2}\label{ggplot2}

\subsubsection{ggplot2基本要素}\label{ggplot2}

\begin{itemize}
\tightlist
\item
  数据(Data)和映射(Mapping)
\item
  几何对象(Geometric)
\item
  标尺(Scale)
\item
  统计变换(Statistics)
\item
  坐标系统(Coordinante)
\item
  图层(Layer)
\item
  分面(Facet)
\item
  主题(Theme)
\end{itemize}

这里将从这些基本要素对ggplot2进行介绍。

\subsection{数据(Data)和映射(Mapping)}\label{datamapping}

\begin{Shaded}
\begin{Highlighting}[]
\KeywordTok{require}\NormalTok{(ggplot2)}
\end{Highlighting}
\end{Shaded}

\begin{verbatim}
## Loading required package: ggplot2
\end{verbatim}

\begin{Shaded}
\begin{Highlighting}[]
\KeywordTok{data}\NormalTok{(diamonds)}
\KeywordTok{set.seed}\NormalTok{(}\DecValTok{42}\NormalTok{)}
\NormalTok{small <-}\StringTok{ }\NormalTok{diamonds[}\KeywordTok{sample}\NormalTok{(}\KeywordTok{nrow}\NormalTok{(diamonds), }\DecValTok{1000}\NormalTok{), ]}
\KeywordTok{head}\NormalTok{(small)}
\end{Highlighting}
\end{Shaded}

\begin{verbatim}
## # A tibble: 6 × 10
##   carat       cut color clarity depth table price     x     y     z
##   <dbl>     <ord> <ord>   <ord> <dbl> <dbl> <int> <dbl> <dbl> <dbl>
## 1  0.71 Very Good     H     SI1  62.5    60  2096  5.68  5.75  3.57
## 2  0.79   Premium     H     SI1  61.8    59  2275  5.97  5.91  3.67
## 3  1.03     Ideal     F     SI1  62.4    57  6178  6.48  6.44  4.03
## 4  0.50     Ideal     E     VS2  62.2    54  1624  5.08  5.11  3.17
## 5  0.27     Ideal     E     VS1  61.6    56   470  4.14  4.17  2.56
## 6  0.30   Premium     E     VS2  61.7    58   658  4.32  4.34  2.67
\end{verbatim}

\begin{Shaded}
\begin{Highlighting}[]
\KeywordTok{summary}\NormalTok{(small)}
\end{Highlighting}
\end{Shaded}

\begin{verbatim}
##      carat               cut      color      clarity        depth      
##  Min.   :0.2200   Fair     : 28   D:121   SI1    :258   Min.   :55.20  
##  1st Qu.:0.4000   Good     : 88   E:186   VS2    :231   1st Qu.:61.00  
##  Median :0.7100   Very Good:227   F:164   SI2    :175   Median :61.80  
##  Mean   :0.8187   Premium  :257   G:216   VS1    :141   Mean   :61.71  
##  3rd Qu.:1.0700   Ideal    :400   H:154   VVS2   : 91   3rd Qu.:62.50  
##  Max.   :2.6600                   I:106   VVS1   : 67   Max.   :72.20  
##                                   J: 53   (Other): 37                  
##      table           price               x               y        
##  Min.   :50.10   Min.   :  342.0   Min.   :3.850   Min.   :3.840  
##  1st Qu.:56.00   1st Qu.:  989.5   1st Qu.:4.740   1st Qu.:4.758  
##  Median :57.00   Median : 2595.0   Median :5.750   Median :5.775  
##  Mean   :57.43   Mean   : 4110.5   Mean   :5.787   Mean   :5.791  
##  3rd Qu.:59.00   3rd Qu.: 5495.2   3rd Qu.:6.600   3rd Qu.:6.610  
##  Max.   :65.00   Max.   :18795.0   Max.   :8.830   Max.   :8.870  
##                                                                   
##        z        
##  Min.   :2.330  
##  1st Qu.:2.920  
##  Median :3.550  
##  Mean   :3.572  
##  3rd Qu.:4.070  
##  Max.   :5.580  
## 
\end{verbatim}

画图实际上是把数据中的变量映射到图形属性上。以克拉(carat)数为X轴变量,价格(price)为Y轴变量。

\begin{Shaded}
\begin{Highlighting}[]
\NormalTok{p <-}\StringTok{ }\KeywordTok{ggplot}\NormalTok{(}\DataTypeTok{data =}\NormalTok{ small, }\DataTypeTok{mapping =} \KeywordTok{aes}\NormalTok{(}\DataTypeTok{x =}\NormalTok{ carat, }\DataTypeTok{y =}\NormalTok{ price))}
\NormalTok{p }\OperatorTok{+}\StringTok{ }\KeywordTok{geom_point}\NormalTok{()}
\end{Highlighting}
\end{Shaded}

\includegraphics{DataAnalysis-8-ggplot2_files/figure-latex/unnamed-chunk-2-1.pdf}
上面这行代码把数据映射XY坐标轴上,需要告诉ggplot2,这些数据要映射成什么样的几何对象,这里以散点为例:

如果想将切工(cut)映射到形状属性。只需要:

\begin{Shaded}
\begin{Highlighting}[]
\NormalTok{p <-}\StringTok{ }\KeywordTok{ggplot}\NormalTok{(}\DataTypeTok{data=}\NormalTok{small, }\DataTypeTok{mapping=}\KeywordTok{aes}\NormalTok{(}\DataTypeTok{x=}\NormalTok{carat, }\DataTypeTok{y=}\NormalTok{price, }\DataTypeTok{shape=}\NormalTok{cut)) }
\NormalTok{p}\OperatorTok{+}\KeywordTok{geom_point}\NormalTok{()}
\end{Highlighting}
\end{Shaded}

\includegraphics{DataAnalysis-8-ggplot2_files/figure-latex/unnamed-chunk-3-1.pdf}
再比如我想将钻石的颜色(color)映射颜色属性:

\begin{Shaded}
\begin{Highlighting}[]
\NormalTok{p <-}\StringTok{ }\KeywordTok{ggplot}\NormalTok{(}\DataTypeTok{data=}\NormalTok{small, }\DataTypeTok{mapping=}\KeywordTok{aes}\NormalTok{(}\DataTypeTok{x=}\NormalTok{carat, }\DataTypeTok{y=}\NormalTok{price, }\DataTypeTok{shape=}\NormalTok{cut, }\DataTypeTok{colour=}\NormalTok{color))}
\NormalTok{p}\OperatorTok{+}\KeywordTok{geom_point}\NormalTok{()}
\end{Highlighting}
\end{Shaded}

\includegraphics{DataAnalysis-8-ggplot2_files/figure-latex/unnamed-chunk-4-1.pdf}

\subsection{几何对象(Geometric)}\label{geometric}

在上面的例子中,各种属性映射由ggplot函数执行,只需要加一个图层,使用geom\_point()告诉ggplot要画散点,于是所有的属性都映射到散点上。
geom\_point()完成的就是几何对象的映射,ggplot2提供了各种几何对象映射,如geom\_histogram用于直方图,geom\_bar用于画柱状图,geom\_boxplot用于画箱式图等等。
不同的几何对象,要求的属性会有些不同,这些属性也可以在几何对象映射时提供,比如上一图,也可以用以下语法来画:

\begin{Shaded}
\begin{Highlighting}[]
\NormalTok{p <-}\StringTok{ }\KeywordTok{ggplot}\NormalTok{(small) }
\NormalTok{p}\OperatorTok{+}\KeywordTok{geom_point}\NormalTok{(}\KeywordTok{aes}\NormalTok{(}\DataTypeTok{x=}\NormalTok{carat, }\DataTypeTok{y=}\NormalTok{price, }\DataTypeTok{shape=}\NormalTok{cut, }\DataTypeTok{colour=}\NormalTok{color))}
\end{Highlighting}
\end{Shaded}

\includegraphics{DataAnalysis-8-ggplot2_files/figure-latex/unnamed-chunk-5-1.pdf}

\subsubsection{直方图}

直方图最容易,提供一个x变量,画出数据的分布。

\begin{Shaded}
\begin{Highlighting}[]
\KeywordTok{ggplot}\NormalTok{(small)}\OperatorTok{+}\KeywordTok{geom_histogram}\NormalTok{(}\KeywordTok{aes}\NormalTok{(}\DataTypeTok{x=}\NormalTok{price))}
\end{Highlighting}
\end{Shaded}

\begin{verbatim}
## `stat_bin()` using `bins = 30`. Pick better value with `binwidth`.
\end{verbatim}

\includegraphics{DataAnalysis-8-ggplot2_files/figure-latex/unnamed-chunk-6-1.pdf}
同样可以根据另外的变量给它填充颜色,比如按不同的切工:

\begin{Shaded}
\begin{Highlighting}[]
\KeywordTok{ggplot}\NormalTok{(small)}\OperatorTok{+}\KeywordTok{geom_histogram}\NormalTok{(}\KeywordTok{aes}\NormalTok{(}\DataTypeTok{x=}\NormalTok{price, }\DataTypeTok{fill=}\NormalTok{cut), }\DataTypeTok{position=}\StringTok{"dodge"}\NormalTok{)}
\end{Highlighting}
\end{Shaded}

\begin{verbatim}
## `stat_bin()` using `bins = 30`. Pick better value with `binwidth`.
\end{verbatim}

\includegraphics{DataAnalysis-8-ggplot2_files/figure-latex/unnamed-chunk-7-1.pdf}
\#\#\# 柱状图
柱状图非常适合于画分类变量。在这里以透明度(clarity)变量为例。按照不同透明度的钻石的数目画柱状图。

\begin{Shaded}
\begin{Highlighting}[]
\KeywordTok{ggplot}\NormalTok{(small)}\OperatorTok{+}\KeywordTok{geom_bar}\NormalTok{(}\KeywordTok{aes}\NormalTok{(}\DataTypeTok{x=}\NormalTok{clarity))}
\end{Highlighting}
\end{Shaded}

\includegraphics{DataAnalysis-8-ggplot2_files/figure-latex/unnamed-chunk-8-1.pdf}
柱状图两个要素,一个是分类变量,一个是数目,也就是柱子的高度。数目在这里不用提供,因为ggplot2会通过x变量计算各个分类的数目。
当然你想提供也是可以的,通过stat参数,可以让geom\_bar按指定高度画图,比如以下代码:

\begin{Shaded}
\begin{Highlighting}[]
\KeywordTok{ggplot}\NormalTok{()}\OperatorTok{+}\KeywordTok{geom_bar}\NormalTok{(}\KeywordTok{aes}\NormalTok{(}\DataTypeTok{x=}\KeywordTok{c}\NormalTok{(LETTERS[}\DecValTok{1}\OperatorTok{:}\DecValTok{3}\NormalTok{]),}\DataTypeTok{y=}\DecValTok{1}\OperatorTok{:}\DecValTok{3}\NormalTok{), }\DataTypeTok{stat=}\StringTok{"identity"}\NormalTok{)}
\end{Highlighting}
\end{Shaded}

\includegraphics{DataAnalysis-8-ggplot2_files/figure-latex/unnamed-chunk-9-1.pdf}

柱状图和直方图是很像的,直方图把连续型的数据按照一个个等长的分区(bin)来切分,然后计数,画柱状图。而柱状图是分类数据,按类别计数。我们可以用前面直方图的参数来画side-by-side的柱状图,填充颜色或者按比例画图,它们是高度一致的。
柱状图是用来表示计数数据的,但在生物界却被经常拿来表示均值,加上误差来表示数据分布,这可以通常图层来实现。

\subsubsection{密度函数图}

说到直方图,就不得不说密度函数图,数据和映射和直方图是一样的,唯一不同的是几何对象,geom\_histogram告诉ggplot要画直方图,而geom\_density则说我们要画密度函数图,在我们熟悉前面语法的情况下,很容易画出:

\begin{Shaded}
\begin{Highlighting}[]
\KeywordTok{ggplot}\NormalTok{(small)}\OperatorTok{+}\KeywordTok{geom_density}\NormalTok{(}\KeywordTok{aes}\NormalTok{(}\DataTypeTok{x=}\NormalTok{price, }\DataTypeTok{colour=}\NormalTok{cut))}
\end{Highlighting}
\end{Shaded}

\includegraphics{DataAnalysis-8-ggplot2_files/figure-latex/unnamed-chunk-10-1.pdf}

\begin{Shaded}
\begin{Highlighting}[]
\KeywordTok{ggplot}\NormalTok{(small)}\OperatorTok{+}\KeywordTok{geom_density}\NormalTok{(}\KeywordTok{aes}\NormalTok{(}\DataTypeTok{x=}\NormalTok{price,}\DataTypeTok{fill=}\NormalTok{clarity))}
\end{Highlighting}
\end{Shaded}

\includegraphics{DataAnalysis-8-ggplot2_files/figure-latex/unnamed-chunk-11-1.pdf}
colour参数指定的是曲线的颜色,而fill是往曲线下面填充颜色。

\subsubsection{箱式图}

数据量比较大的时候,用直方图和密度函数图是表示数据分布的好方法,而在数据量较少的时候,比如很多的生物实验,很多时候大家都是使用柱状图+errorbar的形式来表示,不过这种方法的信息量非常低,这种情况推荐使用boxplot。

\begin{Shaded}
\begin{Highlighting}[]
\KeywordTok{ggplot}\NormalTok{(small)}\OperatorTok{+}\KeywordTok{geom_boxplot}\NormalTok{(}\KeywordTok{aes}\NormalTok{(}\DataTypeTok{x=}\NormalTok{cut, }\DataTypeTok{y=}\NormalTok{price,}\DataTypeTok{fill=}\NormalTok{color))}
\end{Highlighting}
\end{Shaded}

\includegraphics{DataAnalysis-8-ggplot2_files/figure-latex/unnamed-chunk-12-1.pdf}
geom\_boxplot将数据映射到箱式图上,上面的代码,我们应该很熟悉了,按切工(cut)分类,对价格(price)变量画箱式图,再分开按照color变量填充颜色。
ggplot2提供了很多的geom\_xxx函数,可以满足我们对各种图形绘制的需求。
geom\_abline geom\_area\\
geom\_bar geom\_bin2d geom\_blank geom\_boxplot\\
geom\_contour geom\_crossbar geom\_density geom\_density2d\\
geom\_dotplot geom\_errorbar geom\_errorbarh geom\_freqpoly\\
geom\_hex geom\_histogram geom\_hline geom\_jitter\\
geom\_line geom\_linerange geom\_map geom\_path\\
geom\_point geom\_pointrange geom\_polygon geom\_quantile\\
geom\_raster geom\_rect geom\_ribbon geom\_rug\\
geom\_segment geom\_smooth geom\_step geom\_text\\
geom\_tile geom\_violin geom\_vline

\subsubsection{标尺(Scale)}\label{scale}

前面我们已经看到了,画图就是在做映射,不管是映射到不同的几何对象上,还是映射各种图形属性。这一小节介绍标尺,在对图形属性进行映射之后,使用标尺可以控制这些属性的显示方式,比如坐标刻度,可能通过标尺,将坐标进行对数变换;比如颜色属性,也可以通过标尺,进行改变。

\begin{Shaded}
\begin{Highlighting}[]
\KeywordTok{ggplot}\NormalTok{(small)}\OperatorTok{+}\KeywordTok{geom_point}\NormalTok{(}\KeywordTok{aes}\NormalTok{(}\DataTypeTok{x=}\NormalTok{carat, }\DataTypeTok{y=}\NormalTok{price, }\DataTypeTok{shape=}\NormalTok{cut, }\DataTypeTok{colour=}\NormalTok{color))}\OperatorTok{+}\KeywordTok{scale_y_log10}\NormalTok{()}\OperatorTok{+}\KeywordTok{scale_colour_manual}\NormalTok{(}\DataTypeTok{values=}\KeywordTok{rainbow}\NormalTok{(}\DecValTok{7}\NormalTok{))}
\end{Highlighting}
\end{Shaded}

\includegraphics{DataAnalysis-8-ggplot2_files/figure-latex/unnamed-chunk-13-1.pdf}
以数据(Data)和映射(Mapping)一节中所画散点图为例,将Y轴坐标进行log10变换,再自己定义颜色为彩虹色。

\subsubsection{统计变换(Statistics)}\label{statistics}

统计变换对原始数据进行某种计算,然后在图上表示出来,例如对散点图上加一条回归线。

\begin{Shaded}
\begin{Highlighting}[]
\KeywordTok{ggplot}\NormalTok{(small, }\KeywordTok{aes}\NormalTok{(}\DataTypeTok{x=}\NormalTok{carat, }\DataTypeTok{y=}\NormalTok{price))}\OperatorTok{+}\KeywordTok{geom_point}\NormalTok{()}\OperatorTok{+}\KeywordTok{scale_y_log10}\NormalTok{()}\OperatorTok{+}\KeywordTok{stat_smooth}\NormalTok{()}
\end{Highlighting}
\end{Shaded}

\begin{verbatim}
## `geom_smooth()` using method = 'gam'
\end{verbatim}

\includegraphics{DataAnalysis-8-ggplot2_files/figure-latex/unnamed-chunk-14-1.pdf}
这里,aes所提供的参数,就通过ggplot提供,而不是提供给geom\_point,因为ggplot里的参数,相当于全局变量,geom\_point()和stat\_smooth()都知道x,y的映射,如果只提供给geom\_point(),则相当于是局部变量,geom\_point知道这种映射,而stat\_smooth不知道,当然你再给stat\_smooth也提供x,y的映射,不过共用的映射,还是提供给ggplot好。
ggplot2提供了多种统计变换方式: stat\_abline stat\_contour
stat\_identity stat\_summary stat\_bin stat\_density stat\_qq
stat\_summary2d stat\_bin2d stat\_density2d stat\_quantile
stat\_summary\_hex stat\_bindot stat\_ecdf stat\_smooth stat\_unique
stat\_binhex stat\_function stat\_spoke stat\_vline stat\_boxplot
stat\_hline stat\_sum stat\_ydensity
统计变换是非常重要的功能,我们可以自己写函数,基于原始数据做某种计算,并在图上表现出来,也可以通过它改变geom\_xxx函数画图的默认统计参数。

\subsubsection{坐标系统(Coordinante)}\label{coordinante}

坐标系统控制坐标轴,可以进行变换,例如XY轴翻转,笛卡尔坐标和极坐标转换,以满足我们的各种需求。
坐标轴翻转由coord\_flip()实现

\begin{Shaded}
\begin{Highlighting}[]
\KeywordTok{ggplot}\NormalTok{(small)}\OperatorTok{+}\KeywordTok{geom_bar}\NormalTok{(}\KeywordTok{aes}\NormalTok{(}\DataTypeTok{x=}\NormalTok{cut, }\DataTypeTok{fill=}\NormalTok{cut))}\OperatorTok{+}\KeywordTok{coord_flip}\NormalTok{()}
\end{Highlighting}
\end{Shaded}

\includegraphics{DataAnalysis-8-ggplot2_files/figure-latex/unnamed-chunk-15-1.pdf}
而转换成极坐标可以由coord\_polar()实现:

\begin{Shaded}
\begin{Highlighting}[]
\KeywordTok{ggplot}\NormalTok{(small)}\OperatorTok{+}\KeywordTok{geom_bar}\NormalTok{(}\KeywordTok{aes}\NormalTok{(}\DataTypeTok{x=}\KeywordTok{factor}\NormalTok{(}\DecValTok{1}\NormalTok{), }\DataTypeTok{fill=}\NormalTok{cut))}\OperatorTok{+}\KeywordTok{coord_polar}\NormalTok{(}\DataTypeTok{theta=}\StringTok{"y"}\NormalTok{)}
\end{Highlighting}
\end{Shaded}

\includegraphics{DataAnalysis-8-ggplot2_files/figure-latex/unnamed-chunk-16-1.pdf}
这也是为什么之前介绍常用图形画法时没有提及饼图的原因,饼图实际上就是柱状图,只不过是使用极坐标而已,柱状图的高度,对应于饼图的弧度,饼图并不推荐,因为人类的眼睛比较弧度的能力比不上比较高度(柱状图)

\subsubsection{图层(Layer)}\label{layer}

photoshop流行的原因在于PS
3.0时引入图层的概念,ggplot的牛B之处在于使用+号来叠加图层,这堪称是泛型编程的典范。
在前面散点图上,我们已经见识过,加上了一个回归线拟合的图层。
有了图层的概念,使用ggplot画起图来,就更加得心应手。
做为图层的一个很好的例子是蝙蝠侠logo,batman
logo由6个函数组成,在下面的例子中,我先画第一个函数,之后再加一个图层画第二个函数,不断重复这一过程,直到六个函数全部画好。

\begin{Shaded}
\begin{Highlighting}[]
\KeywordTok{require}\NormalTok{(ggplot2)}
\NormalTok{p <-}\StringTok{ }\KeywordTok{ggplot}\NormalTok{(small, }\KeywordTok{aes}\NormalTok{(carat, price))}
\NormalTok{p }\OperatorTok{+}\StringTok{ }\KeywordTok{geom_point}\NormalTok{() }\OperatorTok{+}\StringTok{ }\KeywordTok{geom_smooth}\NormalTok{()}
\end{Highlighting}
\end{Shaded}

\begin{verbatim}
## `geom_smooth()` using method = 'gam'
\end{verbatim}

\includegraphics{DataAnalysis-8-ggplot2_files/figure-latex/unnamed-chunk-17-1.pdf}

\subsubsection{分面(Facet)}\label{facet}

分面可以让我们按照某种给定的条件,对数据进行分组,然后分别画图。
在统计变换一节中,提到如果按切工分组作回归线,显然图会很乱,有了分面功能,我们可以分别作图。

\begin{Shaded}
\begin{Highlighting}[]
\KeywordTok{ggplot}\NormalTok{(small, }\KeywordTok{aes}\NormalTok{(}\DataTypeTok{x=}\NormalTok{carat, }\DataTypeTok{y=}\NormalTok{price))}\OperatorTok{+}\KeywordTok{geom_point}\NormalTok{(}\KeywordTok{aes}\NormalTok{(}\DataTypeTok{colour=}\NormalTok{cut))}\OperatorTok{+}\KeywordTok{scale_y_log10}\NormalTok{() }\OperatorTok{+}\KeywordTok{facet_wrap}\NormalTok{(}\OperatorTok{~}\NormalTok{cut)}\OperatorTok{+}\KeywordTok{stat_smooth}\NormalTok{()}
\end{Highlighting}
\end{Shaded}

\begin{verbatim}
## `geom_smooth()` using method = 'loess'
\end{verbatim}

\includegraphics{DataAnalysis-8-ggplot2_files/figure-latex/unnamed-chunk-18-1.pdf}

\subsubsection{二维密度图}

为了作图方便,我们使用diamonds数据集的一个子集,如果使用全集,数据量太大,画出来散点就糊了,这种情况可以使用二维密度力来呈现。

\begin{Shaded}
\begin{Highlighting}[]
\KeywordTok{ggplot}\NormalTok{(diamonds, }\KeywordTok{aes}\NormalTok{(carat, price))}\OperatorTok{+}\StringTok{ }\KeywordTok{stat_density2d}\NormalTok{(}\KeywordTok{aes}\NormalTok{(}\DataTypeTok{fill =}\NormalTok{ ..level..), }\DataTypeTok{geom=}\StringTok{"polygon"}\NormalTok{)}\OperatorTok{+}\StringTok{ }\KeywordTok{scale_fill_continuous}\NormalTok{(}\DataTypeTok{high=}\StringTok{'darkred'}\NormalTok{,}\DataTypeTok{low=}\StringTok{'darkgreen'}\NormalTok{)}
\end{Highlighting}
\end{Shaded}

\includegraphics{DataAnalysis-8-ggplot2_files/figure-latex/unnamed-chunk-19-1.pdf}

\subsubsection{ggplot2实战}\label{ggplot2}


\end{document}
